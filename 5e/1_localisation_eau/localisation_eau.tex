\documentclass[12pt,a4paper]{article}

%\usepackage[left=1.5cm,right=1.5cm,top=1cm,bottom=2cm]{geometry}
\usepackage[in, plain]{fullpage}
\usepackage{array}
%\usepackage{../../pas-math}
\usepackage{../../moncours}



\input{../../utils_maths}




\date{}
\title{}


\begin{document}
	
	
\chap[num=1, color=red]{Où trouve-t-on de l'eau sur Terre ?}{Olivier FINOT, \today }	

\section{L'eau dans notre environnement \footnotesize{(Activité 1 page 110-111)}}

\begin{myact}{}
	\begin{enumerate}
		\item Les grands réservoirs d'eau visibles sur ces documents sont \kw{les océans} et \kw{la banquise}.
		\item Les palmiers arrivent à pousser en plein désert car \kw{ils ont de faibles besoins en eau}.
		\item Les boissons nous sont nécessaires pour \kw{renouveler l'eau dans notre corps}.
		\item \kw{Le c\oe ur} est l'organe du corps humain qui contient le plus d'eau.
		\item \kw{La peau} est l'organe du corps humain qui contient le moins d'eau.
	\end{enumerate}
\end{myact}

\begin{mybilan}
	Les océans, les fleuves et les glaces polaires sont les grands réservoirs d'eau de la Terre. Tous les êtres vivants contiennent de l'eau, ils en ont besoin pour vivre. \kw{L'eau est présente partout autour de nous, sans elle la vie ne pourrait pas exister}.
\end{mybilan}

\begin{myexos}
	\begin{itemize}
		\item 1 page 118
		\item 9 page 119
		\item 11 page 120
	\end{itemize}
\end{myexos}

\section{Détecter la présence d'eau \footnotesize{Activité 2 page 112}}

\begin{myact}{}
	\begin{enumerate}
		\item Le sulfate de cuivre anhydre est blanc.
		\item Lorsque l'eau rentre en contact avec le sulfate de cuivre anhydre il devient bleu.
		\item Lorsque l'on chauffe le sulfate de cuivre hydraté, il redevient blanc.
		\item L'eau fait changer la couleur du sulfate de cuivre.
		\item Le sulfate de cuivre hydraté change de couleur quand on le chauffe parce l'eau s'évapore.
		\item Le sulfate de cuivre anhydre vire au bleu en présence d'eau.
	\end{enumerate}
\end{myact}

\begin{mybilan}
	En présence d'eau, le \kw{sulfate de cuivre anhydre} blanc, \kw{devient bleu}.
	Lorsque l'on chauffe le sulfate de cuivre hydraté, \kw{l'eau s'évapore, il redevient blanc}.
	Le sulfate de cuivre anhydre est utilisé pour détecter la présence d'eau.
\end{mybilan}

\begin{myexos}
	\begin{itemize}
		\item 3 page 118
		\item 4 page 118
		\item 6 page 119
		\item 7 page 119
		\item 13 page 120
	\end{itemize}
\end{myexos}
\end{document}]