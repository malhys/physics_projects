\documentclass[12pt,a4paper]{article}

%\usepackage[left=1.5cm,right=1.5cm,top=1cm,bottom=2cm]{geometry}
\usepackage[in, plain]{fullpage}
\usepackage{array}
%\usepackage{../../pas-math}
\usepackage{../../moncours}



\input{../../utils_maths}




\date{}
\title{}


\begin{document}
	
	
\chap[num=2, color=blue]{Quel espace un gaz peut-il occuper ?}{Olivier FINOT, \today }	

\section{La forme des liquides et des solides}

\begin{myact}{1 page 124}
	\begin{enumerate}
		\item Non les glaçons n'ont pas la forme du récipient qui les contient.
		\item Le liquide obtenu lorsque les glaçons ont fondu a la forme du récipient.
		\item La surface libre du liquide est plane.
		\item On peut saisir un glaçon avec ses doigts, mais pas de l'eau liquide.
		\item Lorsqu'il est placé dans des récipients de forme différentes, un solide conserve sa forme.
		\item Un solide a une forme propre parce qu'elle ne change pas.
		\item Un liquide placé dans dans des récipients de formes différentes prend la forme de ces récipients.
		\item Le fil à plomb indique la direction verticale, donc le petit côté de l'équerre indique la direction horizontale. On en déduit que la surface libre d'un liquide au repos est horizontale. 
	\end{enumerate}
\end{myact}

\begin{mybilan}
	Un \kw{solide} a une \kw{forme propre} qui ne change pas, on peut le saisir. Un \kw{liquide} prend la \kw{forme du récipient} qui le contient. La surface d'un liquide en contact avec l'air est sa surface libre. Au repos, cette surface libre est \kw{plane et horizontale}.
\end{mybilan}

\begin{myexos}
	\twoCol{
	\begin{itemize}
		\item \exo{5}{133}
		\item \exo{6}{133}
		\item \exo{7}{1332}
	\end{itemize}}
\end{myexos}

\section{Les propriétés des gaz}

\begin{myact}{2 page 125}
	\begin{enumerate}
		\item Lorsque l'eau bout, il se forme de la vapeur dans l'erlenmeyer.
		\item La vapeur d'eau emprisonnée dans l'erlenmeyer occupe tout l'espace disponible.
		\item Quand les deux erlenmeyers sont en communication, on voit apparaître de la buée sur la paroi, car la vapeur est montée dans le deuxième erlenmeyer.
		\item Après la mise en communication, la vapeur occupe l'espace des deux erlenmeyers.
		\item Lorsque l'on appuie sur le piston, le volume d'air contenu dans la seringue fermée diminue.
		\item Non, la vapeur d'eau n'a pas de forme propre.
		\item La vapeur d'eau est expansible car lorsque l'on ajoute le second erlenmeyer, elle l'occupe en plus du premier.
		\item Lorsque l'on appuie sur le piston de la seringue fermée, le volume d'air diminue, l'air est donc compressible.
	\end{enumerate}
\end{myact}

\begin{mybilan}
La \kw{vapeur d'eau} est de l'eau a l'état de \kw{gaz}. Un gaz n'a pas de forme propre, il occupe tout l'espace disponible : il est \kw{expansible}. Un gaz est \kw{compressible}, on peut diminuer sons volume en le comprimant.	
\end{mybilan}


\begin{myexos}
	\begin{multicols}{2}
	
		\begin{itemize}
			\item \exo{2}{132}
			\item \exo{8}{133}
			\item \exo{14}{134}
		\end{itemize}
	
	\end{multicols}
\end{myexos}


\section{Des gaz dans l'eau}

\begin{myact}{3 page 126}
	\begin{enumerate}
		\item Au début de l'expérience le tube à essais contient de l'eau.
		\item Au cours de l'expérience, dans le tube à essais des bulles apparaissent et le niveau de l'eau diminue.
		\item Le bain-marie est à \num{57.6} °C.
		\item Non il n'est pas nécessaire de faire beaucoup chauffer l'eau pétillante pour en récupérer le gaz.
		\item Au cours de l'expérience, l'eau des tubes à essais est remplacée par du gaz.
		\item Le gaz dégagé est récupéré par déplacement d'eau car il prend la place de l'eau contenue dans le tube à essais.
		\item Pour récupérer le gaz contenu dans le l'eau pétillante on peut l'agiter  ou la chauffer.
	\end{enumerate}
\end{myact}

\begin{mybilan}
	L'eau peut contenir des \kw{gaz dissous}. On peut extraire ce gaz de l'eau qui le contient par \kw{agitation} ou par \kw{chauffage}. Le gaz est extrait par \kw{déplacement d'eau}, il prend la place de l'eau contenue dans le tube à essais.
\end{mybilan}

\begin{myexos}
	\begin{multicols}{2}
		
		\begin{itemize}
			\item \exo{}{}
			\item \exo{}{}
			\item \exo{}{}
		\end{itemize}
		
	\end{multicols}
\end{myexos}

\section{Reconnaître le dioxyde de carbone}

\begin{myact}{4 page 127}
	\begin{enumerate}
		\item 
	\end{enumerate}
\end{myact}

\begin{mybilan}
	
\end{mybilan}

\begin{myexos}
	\begin{multicols}{2}
		
		\begin{itemize}
			\item \exo{}{}
			\item \exo{}{}
			\item \exo{}{}
		\end{itemize}
		
	\end{multicols}
\end{myexos}

\appendix

\newpage

\section{Correction des exercices}


\end{document}]