\begin{myact}{4 page 155}
	\begin{enumerate}
		\item L'unité de la température mesurée par la sonde du thermomètre électronique est le degré Celsius (°$C$). \pause
		\item La température du liquide contenu dans le bécher est de \num{17.4} °$C$.\pause
		\item L'intervalle entre deux graduations du thermomètre est alcool correspond à 1 °$C$.\pause
		\item Il faut laisser à la sonde le temps de mesurer précisément la température du liquide, c'est pourquoi on attend que l'affichage se stabilise.\pause
		\item Le réservoir du thermomètre à alcool doit être complètement immergé dans le liquide car c'est lui qui sert à <<mesurer>> la température.\pause
		\item La température mesurée par le thermomètre à alcool est de $17$ °$C$.		
	\end{enumerate}
\end{myact}