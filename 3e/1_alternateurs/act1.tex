\begin{myact}{1 page 168}
	\begin{enumerate}
		\item Lorsque l'aimant et la bobine sont immobiles, le voltmètre indique 0 $V$.\pause
		\item Lorsque l'on approche l'aimant de la bobine, le voltmètre indique \num{20.58} $V$.\pause
		\item Lorsque l'on éloigne l'aimant de la bobine, le voltmètre indique \num{-9.02} $V$.\pause
		\item Une tension apparaît aux bornes de la bobine lorsqu'un aimant approche ou s'éloigne de la bobine.\pause
		\item La signe de la tension produite dépend du sens du mouvement de l'aimant par rapport à la bobine.\pause
		\item Si l'on répète régulièrement ces deux mouvements devant la bobine, une tension variable est produite.
	\end{enumerate}
\end{myact}