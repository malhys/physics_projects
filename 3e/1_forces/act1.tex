\begin{myact}{1 page 84}
	\begin{enumerate}
		\item \begin{itemize}
			\item Dans l'expérience A, le corps étudié est l'objet suspendu au ressort. Lorsqu'il est lâché, une fois le ressort étiré, il est soumis à l'action exercée par le ressort, vers le haut ou vers le bas, ainsi qu'à l'action exercée par la Terre.
			\item Dans l'expérience B, le corps étudié est la limaille de fer. Elle est soumise à l'action exercée par l'aimant et à celle exercée par la Terre (négligeable).
			\item Dans l'expérience C, le corps étudié est l'objet sur lequel est fixé le ballon. Il est soumis à l'action de l'air qui sort du ballon vers la gauche, à l'action de la Terre ainsi qu'à la réaction du support (la table).
			\item Dans l'expérience D, le corps étudié est le glaçon. Il est soumis à l'action exercée par la Terre et à la réaction du support.
		\end{itemize}
		
		\item \begin{itemize}
			\item Les actions de contact sont celles exercées par le ressort, ls supports et l'air expulsé par le ballon.
			\item Les actions à distance sont celles exercées par l'aimant et par la Terre. 
		\end{itemize}
		
		\item 
		
		\item Dans l'expérience C, l'air est expulsé vers la gauche et mets en mouvement le ballon et l'objet auquel il est attaché vers la droite. Le mouvement se fait en réaction (dans le sens contraire) à celui de l'air, on parle de mouvement << à réaction >>.
		
		\item Sur le schéma on voit que la distance parcourue par le glaçon pendant des durées égales reste constante, donc sa vitesse est constante. Son mouvement est rectiligne uniforme.
		
		\item Un corps dans un état initial de repos ou de mouvement et qui est soumis à une nouvelle action, à distance ou de contact, subit alors une mise en mouvement ou la modification de ce mouvement.
	\end{enumerate}
\end{myact}